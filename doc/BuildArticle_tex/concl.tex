\section{Conclusion}

%Quantal response Equilibrium, which consider stochastic behavior, can describe candidates' decisions in the experiment. 

There is variability in the data that can not be explained by random error alone: there are systematic deviations from the standard prediction of NE that can be explained by QRE theory. This theory considers that participants' decisions depends stochasticity on the expected payoffs they face, which also depend on other's decisions that are also random. If we assume that players consider that others behave in the same way, QRE is the point when strategies and beliefs about strategies are the same.  

Considering that citizen-candidate assumption describe well enough the electoral process, the main prediction of QRE theory is that there are more candidates than expected by standard game theory. % review why there is not less entrance
This is the result of a direct and indirect effect of the stochastic best response. % check for direct and indirect effects of stocastic behavior as derivatives using calculus
First, there is a direct effect of the stochasticity that made the Nash winner ($q_{30}$) less probable to enter and increase the probability of others to enter. 
Second, others candidates increase their expected payoffs to participate because the probability of being defeated decreases relative to NE. % there could be the case of a reversal in a Nash winner strategy?, i.e. QRE for her gives a probability of less than 0.5 % review model of federesen specifically the case when there are many candidates at the median and one diferenciated one. In the present case review for the {20,80} equilibrium 
The magnitude of this indirect effect depends on the citizen's position relative to Nash winner; distant candidates to her are more prone to participate. This implication goes according with data observed.

Notice that the model adjust considerably good in al the experimental treatments when the parameter of stochasticity ($\lambda$) is calculated for those games. This parameter is similar in all games except for one where it is relatively higher. The reasons of this increase of $\lambda$ in this game are unclear. It is possible that it is related with the fact that game \textit{ex70} is the only game with an equilibrium where extreme ideal points enter in a higher proportion.

Finally, it most be noted that NE is not a bad forecasting. Nevertheless, QRE not just offer a more precise description, but also predict other interesting phenomena in data.