\chapter*{Appendix}
\addcontentsline{toc}{chapter}{Appendix} % add to index

\section*{Instructions\footnote{Here I present the translation from Spanish of the directions given to the participants.}}



% this appear in ~\Dropbox\Citizen-Candidate\Tratamientos\Treatment 20-30-80

\subsection*{Instructions (read by participants)}

	This is an experiment about decision making in elections. CONACYT has provided funding for this research. The instructions are simple and if you follow them carefully and make good decisions, you can win a CONSIDERABLE MONEY AMOUNT, which will be CASH PAYED to you privately at the end of the session.
	
	After we read the instructions, you will have the opportunity to make your decisions.
	- General Procedure
	
	In this experiment you will have to decide whether or not to compete as a candidate in each of the 30 elections that we will carry out at the end of the instructions.
	
	In each of the elections, one of 3 possible alternatives will be chosen winner by a population of voters (simulated by the computer), according to the procedure of voting that we will see later. The 3 alternatives are represented by positions 20, 30 and 80 located on the next line from 0 to 100 (figure \ref{fig:ideal_points}):
	
\begin{figure}[h]
\centering
\includegraphics[width=0.7\linewidth]{Presentaciones/images/ideal_points}
\caption[Ideal Points]{Ideal Points description}
\label{fig:ideal_points}
\end{figure}

	
	- Group Formation
	
	In each election, you will be part of a group of 3 participants. The composition of each group of participants will change randomly, so that the same group will be composed of different participants in each election. You will never know the identity of who you are participating with.
	
	- Allocation of Alternatives
	
	In each election, one of the alternatives mentioned will be assigned to you as your ideal position. Each participant in your group will be assigned a different position. Thus, a participant will be assigned the 20 position; To another the position 30; And to another, position 80. The allocation of alternatives for each election will be determined in a random manner.
	
	- Candidate Application Procedure
	
	To be considered as a candidate eligible by the voters, you must decide whether or not to post your ideal position in each of the elections. That is, you must decide whether to compete or not to be elected by the voters in each of the elections.
	
	You can only postulate your ideal position and you will not be able to postulate any other position.
	
	Once all the participants have made their decisions to postulate their positions, the winning candidate in each election will be determined according to the voting process described below.
	
	- Procedure for Electing the Winning Candidate
	
	For each election, we have a population of 101 voters. Voters are distributed along the line from 0 to 100 as follows: One voter in each integer represented on the line (figure \ref{fig:ideal_points2}).
	
\begin{figure}[h]
	\centering
	\includegraphics[width=0.7\linewidth]{Presentaciones/images/ideal_points}
	\caption[Ideal Points]{Ideal Points description}
	\label{fig:ideal_points2}
\end{figure} 

	The 101 voting citizens (simulated by the team) will choose the winning candidate, according to the following voting procedure:

\begin{itemize}
	\item \textbf{Plurality Rule:}
	
	1. Each citizen will vote for the candidate closest to his / her position. When there is more than one candidate with the same closeness, the citizen's vote will be randomly assigned to the closest candidates.
	
	2. The winning candidate will be the one who accumulates the highest number of votes. In case of a tie, the winner will be randomly selected among the candidates tied in the first place. Therefore, there will always be only one winner in case there are postulants.
	
	\item \textbf{Run-Off:}
	
	1. In a first round of voting, each citizen will vote for the candidate closest to his or her position. When there is more than one candidate with the same closeness, the citizen's vote will be randomly assigned to the closest candidates.
	
	2. After the first round of voting, the two candidates with the highest number of votes will be selected to participate in a second round of voting.
	
	3. In this second round of voting, each citizen will vote for the candidate closest to his or her position. In case of a tie, the winner will be determined randomly among the tied candidates. Therefore, there will always be only one winner in case there are postulants.
	
	4. If less than three candidates have been nominated, only the first round of voting will be held, with the candidate having the highest number of votes selected.
	
\end{itemize}


- Initial Balance, Profits and Payments

Each participant will start with an initial balance of 140 pesos. At each election, the opening balance will be updated as follows:

In the event that at least one alternative has been postulated:

1. Each participant will be subtracted from the amount in pesos equal to the parameter Alpha (= 0.1) multiplied by the absolute distance between his ideal position and the position of the winning candidate. That is, it will be subtracted from the amount of:

0.1 x | Your Ideal Position - Candidate Position Winner |

2. It will be subtracted to each participant who has decided to postulate his ideal position the amount of C (5 or 20) pesos.

3. The winning candidate will be added the amount of 25 pesos.

In the event that no alternative has been postulated, each participant will be subtracted from the only amount of 40 pesos.

- Accumulated Balance and Payment Procedure

The accumulated balance at the end of each election will be the sum of your initial balance plus the payments and winnings you have earned in each previous election. The balance accumulated at the end of the 30 periods will be paid in closed envelope. If you get a negative balance, you will not get any payment.

- Summary of Instructions

In each election,

1. You will be part of a new group of 3 participants.

2. Each member of the group will be assigned one of the following 3 positions on the line from 0 to 100: 20, 30 and 80.

3. Each participant must decide whether or not to run for election.

4. The 101 voting citizens (simulated by the computer) will determine the winning candidate, voting for the closest to their location.

\begin{itemize}
	\item \textbf{Run-Off}

	A. In the first round, the two candidates with the highest number of votes are elected.
	
	B. In a second round, the winning candidate is chosen from the two candidates with the highest number of votes.
	
	C. Only a first round of voting will be held if the number of nominated candidates is less than three.
	
\end{itemize}

5. The balances will be updated as follows after each election:

A. In the event that at least one alternative has been postulated:

I. It will be subtracted to each participant the amount in pesos equal to 0.1 x | Its Ideal Position - Position of the Winning Candidate |

Ii. It will be subtracted to each participant who has decided to postulate his ideal position the amount of C (5 or 20) pesos.

Iii. The winner will be added the amount of 25 pesos.

B. In the case that no alternative has been postulated, each participant will be subtracted from the only amount of 40 pesos.

6. The balance accumulated at the end of each election will be the sum of your initial balance plus the payments and winnings you obtained in each previous election.

- Factors that influence your earnings in each election

As you can see, your earnings are influenced by three factors:

1. The distance between the chosen winning position and your ideal position.

2. Your decision and the other participants to apply.

3. Be elected winner by voters.

\subsection*{Next Steps (Read by the investigator after reading the instructions)}

Next we will show you the software that we have designed for you to make your decisions. Therefore, leave the instructions on the side of the computer and take the ID LIST sheet that are next to your computer.

Connection with the Server

Each participant must initiate their connection with the server using the following procedure: Enter in the User and Password box the numbers written at the top of their IDENTIFICATION RECORDS formats. Then press the send box.

• Completed records, press end of record.

Screen Reading

Then we will review the information that is now on your screen. In the upper left you will find a column where your USER NUMBER, GROUP SIZE, ROUND NUMBER, TYPE OF ROUND (which in our case we are in the test rounds), and the ACCUMULATED BALANCE (which in Our case is the initial balance of 140 pesos). In the second column on the right side, the value of the ALPHA parameter is indicated, the COST TO BE POSTED, the PAYMENT THAT IS GRANTED FOR WINNING, COST IN THE EVENT OF NO CANDIDATE, and finally NUMBER OF VOTERS.



Practice Rounds

We will now conduct 3 rounds of practice. The primary purpose of these practice rounds is to familiarize you with the software we have designed for you to make your decisions, and therefore will not count toward your payments. If you have any questions during the practice, please raise your hand and I will try to answer them.

Once all have made their decisions, they must wait until all the participants have made their decisions and the computer throw the results of the election.

In case your computer has not activated the box for decision making, it is because we do not have a number divisible by three, so you must wait until the next rounds to activate your screen.

After the practice periods, your initial balance will return to the initial amount of 140 pesos.

Generate Period 1

We now begin the 1st period of practice. (Press: Start Period)

Before they make their decisions see that next is presented a graph where the number of voters in each integer point in the line of 0 to the 100 is indicated, as well as the positions of the different alternatives, including their position. At this level on the right side are some circles where you must dial with the MOUSE to make your decision to run for or not to run.

(After the computer generates the results)

In the Results Table, a chart is indicated on the right side indicating the number of votes obtained by each of the candidates who ran. In case of no graphic display, it means that none of the participants in the group ran. On the left side, the amount of the initial balance, the payment to win, which will be greater than zero if you have won the election, the cost of having been nominated, which will be greater than zero if you decided to apply, the cost per Its distance from the winner, the cost for lack of candidates, which will be greater than zero in case no member of the group has decided to run, and finally the accumulated balance, which is the sum of their initial balance plus payments minus Costs.

Do you have questions?

Results Sheet: If you wish to keep a record of your accumulated balance, please use the back of the information record sheet. In any case, the computer will be recording its decisions and accumulated payments.

Generate Period 2

Let us proceed now to the 2nd period of practice. (Press: Start Period) Proceed to make your decisions.

Do you have questions?

Generate Period 3

Let's proceed to 3rd. Period of practice. (Press: Start Period) Proceed to make your decisions.

Do you have questions?

Periods Actual or Played by Money

Now we will proceed to carry out the 30 periods to be played for money. Your initial balance will return to the initial amount of 140 pesos.

• Check all screens.

After the experiment begins, you are not allowed to speak or communicate with other participants. Otherwise, we will be forced to exclude it from the experiment. Please concentrate on your computer screen. If you have any questions, please raise your hand and one of us will approach you and we will try to answer it.

Generate Period 1-30 (Press: Start of Period)
Proceed to make their decisions.

Final payment

Your payment is the amount that appears on the balance of your screen.

Please stay in your posts. One of us will give you a final questionnaire and your payment receipt to be filled out by you. Please add up the balances obtained in both parts of the session.

Then they will be called out to receive their payment. Please go to leave all the material that was given to them.

Thank you very much for your participation!


