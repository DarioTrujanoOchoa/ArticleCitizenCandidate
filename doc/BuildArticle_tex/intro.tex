\section{Introduction}

% why is it worth to know
In any democracy, politicians have to decide if they run for office before elections. We suppose they consider payoffs and chances to win which, ultimately, depend on electoral preferences and who their adversaries are going to be. 
However, few people would affirm that politicians are completely predictable, i.e., choosing always the more profitable option, as assumed in Nash equilibrium (NE) calculations.
If we want to understand and predict who will campaign, we need a model of how the political process works and a theory that convey more realistic assumptions of candidates. 
In order to do this, the Citizen-Candidate model proposed by \citeA{Osborne1996} was evaluated experimentally and the results analyzed with the QRE \cite{McKelvey1995,Goeree2016} which relax the perfect maximizing assumption.

This work is located at the intersection of two perspectives:
first, Behavioral Game Theory \cite{Camerer2003a, Gachter2004}, from where QRE was proposed; 
and second, Political Economy \cite{Besley2007}, where the citizen-candidate model is an important instance.
%The implementation of QRE to refine the predictions of the citizen candidate model is the theoretical innovation achieved in this thesis.
%Finally, an important trait to note is the experimental which is the methodological approach used in this thesis. %experimental literature
%I conclude that, considering stochastic and non-perfect maximizing decision rule, it is possible to better describe candidates' decisions in the experiment.

%relation with others empirical studies in political economics
In political economics, it is more common to study voters' behavior \cite{Besley2007}, and candidates' behavior is frequently studied to test median voter theorem under different assumptions \cite{Palfrey2009}. 
Hotelling-Downs model is one the main formal tools to model candidates behavior. In the classical setting, two political options are located on an interval that represents voters' preferences. Each voter chooses the closest candidate to them. Thus, candidates do better as closer to each other and move towards the median if standing in the same position seeking to lure more votes.
In contrast with this model, the Citizen-Candidate model proposed by \citeA{Osborne1996} considers candidate's ideal policy and, once in office, they choose their preferred policy. This model has been empirically implemented by \citeA{Cadigan2005, Elbittar2009}. 
They compared the results with NE and concluded that observed behavior approaches to it, although there are facts that can not be understood under this framework. %Considering more realistic assumptions about behavior lead to more precise description and understanding of equilibrium outcomes seen in other experiments \cite{Goeree2003}, and these results can be applied to real settings with more reliance.

% methodological justification
Elections are clearly modeled with game theory; rules are explicit for voters and candidates.
The objective of political competition models is to determine who will campaign, predict the winner, and the public policies implemented, i.e., NE of the election game.
The experimental approach is advisable to evaluate the effect of relevant variables, and consider the structure of the political process.
Observational data is frequently noisy and/or do not show variance in the key parameters of the model. Particularly, the voting rules are not commonly changed. For this reason, experiments are now part of the basic tools in political economics investigation \cite{Palfrey2009}.

In the experiment, costs of entry, different electoral systems and set of ideal points were modified to evaluate the predictions of the model.
The experimental implementation was exactly the same as that used by \citeA{Elbittar2009} who also analyzed two of the six games presented here without adjusting the QRE.

The Citizen-Candidate model assumes voters' preferences across an unidimensional line in $\mathbb{R}$, as the Hotelling-Downs model.
The decision of campaign depends on the campaign cost, winning benefits, the voting system and the specific set of possible candidates. 
The general model allows any citizen to participate in the competition. 
However, in the experiment, only a finite subset of citizens have this possibility. \footnote{This can be due to barriers, v.g., highly enough costs for the others citizens.} 
The prediction of this model is behavior in line with NE. However, there is great variability within the responses of each individual and a tendency to over-participate was reported by \citeA{Elbittar2009}. 
In the same article, Quantal Response Equilibrium was proposed as a possible tool to explain this phenomenon.

% citizen-candidate model
%Elections entail a number of citizens choosing who will make the decisions among a diversity of candidates and campaign promises, 
%and there are different determinants of participation: cost of campaign, benefits of being elected, characteristics of the candidates, and different electoral systems.

% QRE description
Frequently, we observe others behaving as if randomly and make decisions even against their own benefit. 
We should concern about this during interactions, when others' decisions impact our own welfare. 
In consequence, we must take into account the possibility of others' mistakes and our own behavior. 
However, decisions are not as erratic as they seem: 
we expect others -and ourselves- to commit less mistakes when payoffs are bigger. 
%Also, it is expected people to be more prone to choose better options, even when there is always the possibility that they choose the worst.
These characteristics are captured by QRE in what is called a regular response function (or stochastic best response) which states a mixed strategy profile: 
probabilities of every possible action to be chosen given their expected payoffs. 
A regular response function generalizes best response correspondence in classical game theory, and the concept of equilibrium is completely analogous: 
a point where beliefs and strategies coincide for every player.

% conceptual relevance
The QRE theory considers that agents have certain level of rationality: they are not perfect maximizers, but they are strategic, i.e., they think about others' actions and randomness. 
In this theory, NE is a especial case with perfectly maximizers agents. % citation 
Evaluation of QRE in the context of elections presents an opportunity to check for robustness of the theory and its assumptions, 
whereas providing a more reliable prediction of elections. 



%Elections is the more characteristic trait of any democracy. Many person in the world has been in contact with this practice\footnote{49.3\% of the world population live in a full or flawed democracy according to the Democracy Index}. 
%Furthermore, it is common that in public and private organization some type of democratic process took place time to time. 
%Considering the impact that this kind of process have over the social welfare, a better understanding of democracies seems important and interesting.

%Therefore, the other important objective of this thesis is to explore whether the QRE model can explain the behavior observed in the experiment. Although, it is important to mention that this model is based on the comparison between expected utilities of the options which requires an utility function that, if not linear, affects the predictions. \citeA{Goeree2003}, and \citeA{TrujanoOchoa2013a} found that considering a concave utility function improves the fit of the model.

%main results consistent with conclusions

% Sonia's comments

%Resaltar la importancia del problema de investigación
%Buscar definición de democracia de otra fuente
%RElavancia de estudiar por qué el ciudadano decide entrar a competir para UN PUESTO DE ELECCIÓN POPULAR
%Dejar clara la pregunta de investigación
%relavancia y problemas del modelo de citizen-candidate
%relevancia del fenomeno observado (costos de entrada, quiza perdida de bienestar?)
%



