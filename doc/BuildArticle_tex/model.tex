\section{Model}

%The Citizen-Candidate (Osborne and Slivinski 1996) model of political competition is described in detail.

The Citizen-Candidate model assumes that preferences can be represented
in the real line following the Hotelling-Downs's location model. It is
common to normalize preferences in the \([0,1]\) interval but, according with the
empirical setup of the model, the interval \(A=[0,100]\) was considered.
The empirical model was implemented in this fashion: the discrete subset
\(Q={q_1, ..., q_n}, q_i \in A\) represents ideal policies, and each
citizen can be referred to by their ideal which is indicated
exogenously.

Possible candidates consider the cost of participation \((c)\), the
possible benefits of being elected or ego rent \((b)\), and their
preferences over the possible final policies implemented. In order to
model these considerations, equation \ref{eq:Utility} represent the preferences of citizens:

\begin{equation}
u_i(x,q_i)=
\begin{cases}
-D, & \text{if } s_i=0, \forall i \in Q \\
-\alpha||x-q_i|| - cs_i + bw_i(s), & \text{otherwise} 
\end{cases}\label{eq:Utility}
\end{equation}

The parameter \(\alpha\) indicates the importance of the final policy
chosen. Citizen lose utility proportionally to the distance from their ideal point (\(q_i\)).
Thus, citizens prefer policies closer to their locations. The variables
\(s_i \in \{0,1\}\) and \(w_i \in \{0,1\}\) stand for the entry decision
(\(entry=1\), $not entry = 0$), and the final result (\(win=1\), $lose=0$), respectively. Notice
that winning depends on the voting system and the profile of decision
made by citizens (\(s \in \Pi s_i\)), which defines the candidates set.

\subsection{Electoral Systems}\label{electoral-systems}

In this model, sincere voters are considered. Thus there is not the possibility to create coalitions, and each candidate get the voters closest to them.  

\begin{itemize}
	\item \textbf{Plurality Rule}:
	In this System, the candidate who get more votes, calculated as in the Hotelling-Downs's location model, is elected. 
	In the case of a tie the winner is chosen randomly as stated before.
	\item \textbf{Runoff}:
	This system take the two candidates with highest votes from a first round, and then a second round of voting is held only with this two options. In the case that one candidate got more that a half of the votes in the first round, she wins without a second round. Winner is chosen randomly if there is a tie in the second round.
\end{itemize}

\subsection{Stages}\label{stages}

Following (Besley and Coate 1997) the three stages of the elections (entry, voting and policy choice) are
presented in inverse order. This clarifies the subsequent Nash Equilibria
calculus as a consequence of backward induction reasoning.

\subsubsection{Policy Choice}\label{policy-choice}

Once having won and holding the office, winner candidate (\(q_i^*\)) can
implement any policy. It is clear that any
winner will choose their ideal policy (\(x=q_i^*\)) maximizing equation
\ref{eq:Utility}. Therefore, we can define
\(x: q \in Q \rightarrow [0,100]\).
It will be clear that \(x\) is a random variable while \(w\) is drawn
from a set of winner candidates.

An important assumption is perfect information: all players and voters know the
preferences of others which means that candidates cannot lie about
the policy they is going to implement once in office. 
These assumptions, contrast with the classical
Hotelling-Downs model where candidates freely move along the preferences and are enforced to keep their campaign promises.

\subsubsection{Voting}\label{voting}

Given the subset of citizens who have decided to participate in the
election (\(C \subseteq Q\)), there is a voting system that designates
the winners set (\(W: C \rightarrow C\)). Plurality Rule (PR) or Run-off
(RO) determine function \(W\). In both systems, when two or more
candidates get the same number of votes, and more than others, they
belong to \(W\) from were \(q*\) is randomly sampled. In both cases, it
is assumed that the distribution of voters over \(A\) is uniform.
Citizens vote sincerely and choose the closer candidate to their own
location. Furthermore, given that there is a continuum of voters, each
citizen represent a single point in \(A\), then their own vote choice is
negligible\footnote{Remember that there is a continuum of voters from
	which only a discrete subset of them could be candidates referred to
	as citizens.}. The assumption of sincere vote does not allow for
coordination between voters (check Federssen et al 1990).

\subsubsection{Entry}\label{entry}

Each citizen decides simultaneously whether to campaign. They do so
thinking strategically: considering what other citizens would do
evaluating expected payoffs and also thinking strategically. Therefore,
the resulting set \(C\) is a NE. %Now I continue with the construction of expected utility function.

In the case where no citizen presents a candidacy, everyone receives a
payoff of \(-D\) which is high enough to deter this case from the
equilibrium set. Also, note that the cardinality of \(W(C)\) could be more than 1
(\(\#((W(C)) \ge 1\)). In such a case, the winner is chosen with equal
probability due to sincere voting. Then, the citizen \(i\)'s probability
of being elected is defined: \(P_i(C)=1/\#((W(C))\) if \(q_i \in W(C)\)
and \(0\) otherwise. Therefore, the winner is a random variable.

In consequence, given the vector of parameters
\(\theta= (\alpha, c, b, D)\), the expected utility for each citizen can
be defined.

\begin{equation}
U_i( C;\theta )=\Sigma P_j(C)(u_i(x=q_j,q_i;\theta))
\end{equation}

\label{eq:EU}

Because the candidates set is defined from the preceding entry decision
(\(C=C(S)\)), it is convenient to write the expected value as a function
of the profiles:

\[
U_i(s), s \in \Pi_{i \in N} S_i
\]

Now, lets define the best response correspondence as: 
$$BR_i(s)=argmax_{S_i}\{U_i(s)\}$$

From this equation, the existence of a NE can be shown in a standard way
using the fixed point theorem. Considering the points $(s*)$ where $(BR_1(s*),..., BR_n(s*)) = s*$

\section{Experimental Design}\label{experimental-settings-games}

The six experimental treatments had three candidates with different ideal points from the interval $[0, 100]$ which represents the sincere voters.
The six different games were built from 2 entry costs, 2 voting rules and 3 ideal points sets.
Table \ref{tab:parameters} summarizes each game parameters. 
The last three columns show the ideal points labeled \emph{Left}, \emph{Center} and \emph{Right} according with the relative position held by each candidate. The last four game have the same ideal position and were built from crossing costs $5$ and $20$ with the two voting rules. 

\begin{table}[!htbp] \centering 
	\caption{Parameters in the six treatments} 
	\label{tab:parameters} 
	\begin{tabular}{@{\extracolsep{5pt}} ccccccccc} 
		\\[-1.8ex]\hline 
		\hline \\[-1.8ex] 
		Games & VotingRule & alpha & costs & benefit & D & Left & Center & Right \\ 
		\hline \\[-1.8ex] 
		ex70 & Plurality Rule & $0.100$ & $5$ & $25$ & $40$ & $30$ & $50$ & $70$ \\ 
		ex80 & Plurality Rule & $0.100$ & $5$ & $25$ & $40$ & $30$ & $50$ & $80$ \\ 
		PR\_LC & Plurality Rule & $0.100$ & $5$ & $25$ & $40$ & $20$ & $30$ & $80$ \\ 
		PR\_HC & Plurality Rule & $0.100$ & $20$ & $25$ & $40$ & $20$ & $30$ & $80$ \\ 
		RO\_LC & Run-Off & $0.100$ & $5$ & $25$ & $40$ & $20$ & $30$ & $80$ \\ 
		RO\_HC & Run-Off & $0.100$ & $20$ & $25$ & $40$ & $20$ & $30$ & $80$ \\ 
		\hline \\[-1.8ex] 
	\end{tabular} 
\end{table} 

\subsection{Participants}

For each game there were three sessions with different participants who were students of various undergraduate programs at ITAM in Mexico City. They played three practice trials at the beginning of each session and at most during 30 effective trials.
Participants were randomly matched and assigned an ideal point each trial. 
Less trials happened if participants lost all the money given or because there were a non-multiple of three; then, one or two participants waited until the next trial-match. 
Each one initiated with $\$140$, this amount changed through the session according with the payoffs. Participants were allowed to continue until they finished a trial with negative balance. 

\subsection{Nash Equilibria}

There are only two possible equilibria:

\begin{itemize}
	\item One-Candidate Equilibrium:
		$s_i=1, s_{-i}=0$
	\item Two-Candidate Equilibrium:
		$s_i=s_j=1, i \neq j, s_{k}=0, \forall k\neq i,j$
\end{itemize}

The Nash equilibria in pure strategies are stated in table \ref{tab:equilibria}. The only games with two-candidate equilibrium are \emph{ex70} and \emph{PR\_LC}, which are those with low cost, extreme ideal points symmetric around median and voting  plurality rule. The closest equilibrium to empirical proportion of entering are marked with an asterisk. 

\begin{table}[!htbp] \centering 
	\caption{Possible equilibria with the ideal points of entering candidates}
	\label{tab:equilibria} 
	\begin{tabular}{@{\extracolsep{5pt}} ccc} 
		\\[-1.8ex]\hline 
		\hline \\[-1.8ex] 
		Games & OneCandidate & TwoCandidate \\ 
		\hline \\[-1.8ex] 
		ex70 & $50$ & \textbf{30*, 70*} \\ 
		ex80 & \textbf{50*} &  \\ 
		PR\_LC & \textbf{30*} & $20$, $80$ \\ 
		PR\_HC & \textbf{30*} &  \\ 
		RO\_LC & \textbf{30*} &  \\ 
		RO\_HC & \textbf{30*} &  \\ 
		\hline \\[-1.8ex] 
	\end{tabular} 
\end{table} 

\subsection{QRE}\label{qre}

The Quantal Response Equilibrium (QRE) proposed by (Richard D McKelvey
and Palfrey 1995) is constructed on the base of a stochastic best
response function. The most used implementation is the logistic function
over the difference between the expected payoff of the options:

\begin{equation}\label{fn:SBR}
\sigma_{ij}= \displaystyle\frac{e^{(\pi_{j})\lambda}}{e^{(\pi_{j})\lambda}+e^{(\pi_{i\ne j})\lambda} } 
=\displaystyle\frac{1}{1+e^{(\pi_{i\ne j}-\pi_{j})\lambda }}        
\end{equation}

The degree of stochasticity in the electionis determined by \(\lambda\). When
\(\lambda \rightarrow 0\) the election is a fair coin tossing for each
option independent of the expected payoffs (minimum level of
rationality), and when \(\lambda \rightarrow \infty\) the election is
deterministic with the more profitable action being chosen with
certainty (complete rationality). The expected payoff of action \(j\) is
represented by \(\pi_j\). Notice that $\pi_j: \sigma_{-i} \rightarrow \mathbb{R}$, where $\sigma_{-i}$ stands for others' distribution of probability.

An easy way to visualize the effect of \(\lambda\) over decisions is to
remember the logistic regression: the dependent variable is the
probability to choose option \(j\), and the independent variable is the
difference between expected payoffs of the two options. Intercept is
fixed in zero -which imply equal probability when options' expected
payoffs are the same-, and slope is precisely \(\lambda\), what
determines how step is the logistic function. For this reason,
\(\lambda\) is also interpreted as the level of rationality: as it
increases, decision will be better in terms of expected payoffs and less
uncertain.


With equation \ref{fn:SBR} a stochastic best response is defined:

\begin{equation}
\sigma^*_{ij}(\lambda, \beta, \pi_{j}(\sigma_{-i}))
\end{equation}\label{eq:SBR} 

, which is the player \(i\)'s probability of choose the
actions \(j\) (i.e.~Entry or Not Entry in the Citizen-Candidate model).
Vector \(\beta\) commonly refers to economic parameters that can measure
risk aversion or altruism. In the present analysis, I do not consider
these possibilities, but because \(\beta\) refers to changes in the
original payoff matrix, I will use \(\beta\) to refer to different
games.

With equation \ref{eq:SBR}, a distribution over actions for each player
is defined: \[\sigma^*_{i}(\lambda, \beta, \sigma_{-i})\]

This stochastic best response have the proprieties of a regular quantal
response function stated by (Goeree, Holt, and Palfrey 2016):

\begin{itemize}
	\item Interiority: $\sigma_{ij} >0$
	\item Continuity: $\sigma_{ij}$ is continuous and differentiable. 
	\item Responsiveness:  $\partial \sigma_{ij} / \partial \pi_{ij} >0 $
	\item Monotonicity: $\pi_{ij}> \pi_{ik} \Rightarrow \sigma_{ij}>\sigma_{ik}$
\end{itemize}

Notice that stochastic best response is a function of others' mixed
strategies (\(\sigma_{-i}\)). This is the case because, as seen in
equation \ref{fn:SBR}, probability depends on \(\pi_{j}\) which is a
function of others' strategies: \(\pi_{j}(\sigma_{-i})\). In the case of
the Citizen-Candidate model, \(\pi_j\) is the expected payoff of
equation \ref{eq:EU}.

Define then the function:

\begin{equation}\label{eq:sigma}
\sigma = (\sigma^*_{1}, ..., \sigma^*_{N})
\end{equation}

Then, quantal response equilibrium (\(\sigma^*\)) is a fixed point of
equation \ref{eq:sigma}, that now is a function only of \(\lambda\) and
\(\beta\). Due to the proprieties of the stochastic best response, and
using the fixed point theorem, the equilibrium existence is assured.
The QRE can be seen as the solution of a non-linear system of equations which is solved numerically. 

Note that a different equilibrium is predicted depending of the value of
\(\lambda\) and that players tend towards pure strategies as \(\lambda\)
goes larger. This equilibrium is called logit equilibrium (Goeree, Holt,
and Palfrey 2016). The software Gambit (Richard D. McKelvey, McLennan,
and Turocy 2014) allows to see those equilibria as a function of the
rationality parameter \(\lambda\). I realized the calculus in \emph{R}
(R Core Team 2017) with the package \emph{nleqslv} (Hasselman 2017), and
compare the result with gambit's; they are the same.

