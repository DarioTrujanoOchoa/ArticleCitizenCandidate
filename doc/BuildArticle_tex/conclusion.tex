\chapter{Conclusion}

The quantal response Equilibrium (QRE) theory, which consider stochastic behavior, can describe candidates' decisions in the experiment. This allows more reliance when applying the citizen-candidate model to real settings.

There is variability in the data that can not be explained just by random error: there are systematic deviations from the standard prediction of NE. 
These phenomena can be explained by QRE theory, considering participants' decisions depends stochasticity on the expected payoffs they face in the game, which in turn depend on other's decisions that are also random, and at the same time they consider that others behave in the same way. %implications.  

%The QRE goes according with the idea that people are stochastic but still payoff influenced (according with the proprieties of the regular quantal response function)

%The final phenomenon I analyze is candidate behavior. 
Considering that citizen-candidate assumption describe well enough the electoral process, the main prediction of QRE theory is that there are more candidates than expected by standard game theory. % review why there is not less entrance
This is the result of a direct and indirect effect of the stochastic best response. % check for direct and indirect effects of stocastic behavior as derivatives using calculus
First, there is a direct effect of the stochasticity that made the Nash winner ($q_{30}$) less probable to enter and increase the probability of others to enter. 
Second, others candidates increase their expected payoffs to participate because the probability of being defeated decreases relative to NE. % there could be the case of a reversal in a Nash winner strategy?, i.e. QRE for her gives a probability of less than 0.5 % review model of federesen specifically the case when there are many candidates at the median and one diferenciated one. In the present case review for the {20,80} equilibrium 
The magnitude of this indirect effect depends on the citizen's position relative to Nash winner; distant candidates to her are more prone to participate. This implication goes according with data observed.

In the experiment, citizen $q_{80}$ was more prone to enter than $q_{20}$. 
The effect of over participation is expected to be bigger in isolated candidates because they are already losing: the Nash winner is far from them. Meanwhile, candidates closer to the Nash winner have less to win if they are elected. 
This is an implication from the sincere voting assumption because closer candidates compete for the same voters.%put formally 

The same logic goes for the analysis over the effects of other variables. Costs and election systems affect the expected payoffs of enter the competition. High cost and Run-off system decrease entrance of Nash losers. 
The effect of the last variable is due to all participants campaign, $q_{30}$ wins in Run-off system, and $q_{80}$ in Plurality Rule, $q_{20}$ always lose. 
%This explained the statistically significant effect of both variables on $q_{80}$ but not on $q_{20}$. % check for expected payoffs, theoretically and empirically. stochasticity is better for whom? 

Additionally, an analogous result to median voter theorem was found: the closest candidate to the median (i.e. $q_{30}$) entering alone was always an equilibrium. % check for generality % congeture: QRE is always a deviation from closest median equlibrium, i.e. it is the logit equilibrium.

There are no important differences in $\lambda$ between global and by game analysis; it seems improbable -and also inconvenient- that the level of rationality ($\lambda$) varies across game, which would mean that it is a function of payoffs. A better adjust could be reached if non linear utility function were considered, as previously done by \citeA{Goeree2003} and \citeA{TrujanoOchoa2013a}. But in this case, it is more difficult to implement due to random choice when ties, and goes further the objective of this thesis. Although, it seems plausible that differences between global and by game analysis would shorten. 
%This approach has been used before and preserves the falsifiability of the model.

There is individual heterogeneity, but in general all participants can be described by a single distribution. From a parsimony argument it is better to consider that all participants share the same level of rationality measured with the parameter $\lambda$. 
Nevertheless, this heterogeneity can be considered by multilevel statistical models. These models are complex to calculate but include the assumption that participants come from the same distribution. 
Bayesian Statistics analysis could be an adequate method: the inferences would not be on a specific parameter, but over the distribution of $\lambda$ in the population.


Finally, it most be noted that NE is not a bad forecasting. Nevertheless, QRE not just offer a more precise description, but also predict other interesting phenomena in data.